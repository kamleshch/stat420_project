\documentclass[]{article}
\usepackage{lmodern}
\usepackage{amssymb,amsmath}
\usepackage{ifxetex,ifluatex}
\usepackage{fixltx2e} % provides \textsubscript
\ifnum 0\ifxetex 1\fi\ifluatex 1\fi=0 % if pdftex
  \usepackage[T1]{fontenc}
  \usepackage[utf8]{inputenc}
\else % if luatex or xelatex
  \ifxetex
    \usepackage{mathspec}
  \else
    \usepackage{fontspec}
  \fi
  \defaultfontfeatures{Ligatures=TeX,Scale=MatchLowercase}
\fi
% use upquote if available, for straight quotes in verbatim environments
\IfFileExists{upquote.sty}{\usepackage{upquote}}{}
% use microtype if available
\IfFileExists{microtype.sty}{%
\usepackage{microtype}
\UseMicrotypeSet[protrusion]{basicmath} % disable protrusion for tt fonts
}{}
\usepackage[margin=1in]{geometry}
\usepackage{hyperref}
\PassOptionsToPackage{usenames,dvipsnames}{color} % color is loaded by hyperref
\hypersetup{unicode=true,
            pdftitle={UK Household Annual Expenditure Analysis},
            pdfauthor={STAT 420: Final Project, Summer 2019, kamlesh2, kdp3, pa9},
            colorlinks=true,
            linkcolor=Maroon,
            citecolor=Blue,
            urlcolor=cyan,
            breaklinks=true}
\urlstyle{same}  % don't use monospace font for urls
\usepackage{color}
\usepackage{fancyvrb}
\newcommand{\VerbBar}{|}
\newcommand{\VERB}{\Verb[commandchars=\\\{\}]}
\DefineVerbatimEnvironment{Highlighting}{Verbatim}{commandchars=\\\{\}}
% Add ',fontsize=\small' for more characters per line
\usepackage{framed}
\definecolor{shadecolor}{RGB}{248,248,248}
\newenvironment{Shaded}{\begin{snugshade}}{\end{snugshade}}
\newcommand{\AlertTok}[1]{\textcolor[rgb]{0.94,0.16,0.16}{#1}}
\newcommand{\AnnotationTok}[1]{\textcolor[rgb]{0.56,0.35,0.01}{\textbf{\textit{#1}}}}
\newcommand{\AttributeTok}[1]{\textcolor[rgb]{0.77,0.63,0.00}{#1}}
\newcommand{\BaseNTok}[1]{\textcolor[rgb]{0.00,0.00,0.81}{#1}}
\newcommand{\BuiltInTok}[1]{#1}
\newcommand{\CharTok}[1]{\textcolor[rgb]{0.31,0.60,0.02}{#1}}
\newcommand{\CommentTok}[1]{\textcolor[rgb]{0.56,0.35,0.01}{\textit{#1}}}
\newcommand{\CommentVarTok}[1]{\textcolor[rgb]{0.56,0.35,0.01}{\textbf{\textit{#1}}}}
\newcommand{\ConstantTok}[1]{\textcolor[rgb]{0.00,0.00,0.00}{#1}}
\newcommand{\ControlFlowTok}[1]{\textcolor[rgb]{0.13,0.29,0.53}{\textbf{#1}}}
\newcommand{\DataTypeTok}[1]{\textcolor[rgb]{0.13,0.29,0.53}{#1}}
\newcommand{\DecValTok}[1]{\textcolor[rgb]{0.00,0.00,0.81}{#1}}
\newcommand{\DocumentationTok}[1]{\textcolor[rgb]{0.56,0.35,0.01}{\textbf{\textit{#1}}}}
\newcommand{\ErrorTok}[1]{\textcolor[rgb]{0.64,0.00,0.00}{\textbf{#1}}}
\newcommand{\ExtensionTok}[1]{#1}
\newcommand{\FloatTok}[1]{\textcolor[rgb]{0.00,0.00,0.81}{#1}}
\newcommand{\FunctionTok}[1]{\textcolor[rgb]{0.00,0.00,0.00}{#1}}
\newcommand{\ImportTok}[1]{#1}
\newcommand{\InformationTok}[1]{\textcolor[rgb]{0.56,0.35,0.01}{\textbf{\textit{#1}}}}
\newcommand{\KeywordTok}[1]{\textcolor[rgb]{0.13,0.29,0.53}{\textbf{#1}}}
\newcommand{\NormalTok}[1]{#1}
\newcommand{\OperatorTok}[1]{\textcolor[rgb]{0.81,0.36,0.00}{\textbf{#1}}}
\newcommand{\OtherTok}[1]{\textcolor[rgb]{0.56,0.35,0.01}{#1}}
\newcommand{\PreprocessorTok}[1]{\textcolor[rgb]{0.56,0.35,0.01}{\textit{#1}}}
\newcommand{\RegionMarkerTok}[1]{#1}
\newcommand{\SpecialCharTok}[1]{\textcolor[rgb]{0.00,0.00,0.00}{#1}}
\newcommand{\SpecialStringTok}[1]{\textcolor[rgb]{0.31,0.60,0.02}{#1}}
\newcommand{\StringTok}[1]{\textcolor[rgb]{0.31,0.60,0.02}{#1}}
\newcommand{\VariableTok}[1]{\textcolor[rgb]{0.00,0.00,0.00}{#1}}
\newcommand{\VerbatimStringTok}[1]{\textcolor[rgb]{0.31,0.60,0.02}{#1}}
\newcommand{\WarningTok}[1]{\textcolor[rgb]{0.56,0.35,0.01}{\textbf{\textit{#1}}}}
\usepackage{longtable,booktabs}
\usepackage{graphicx,grffile}
\makeatletter
\def\maxwidth{\ifdim\Gin@nat@width>\linewidth\linewidth\else\Gin@nat@width\fi}
\def\maxheight{\ifdim\Gin@nat@height>\textheight\textheight\else\Gin@nat@height\fi}
\makeatother
% Scale images if necessary, so that they will not overflow the page
% margins by default, and it is still possible to overwrite the defaults
% using explicit options in \includegraphics[width, height, ...]{}
\setkeys{Gin}{width=\maxwidth,height=\maxheight,keepaspectratio}
\IfFileExists{parskip.sty}{%
\usepackage{parskip}
}{% else
\setlength{\parindent}{0pt}
\setlength{\parskip}{6pt plus 2pt minus 1pt}
}
\setlength{\emergencystretch}{3em}  % prevent overfull lines
\providecommand{\tightlist}{%
  \setlength{\itemsep}{0pt}\setlength{\parskip}{0pt}}
\setcounter{secnumdepth}{0}
% Redefines (sub)paragraphs to behave more like sections
\ifx\paragraph\undefined\else
\let\oldparagraph\paragraph
\renewcommand{\paragraph}[1]{\oldparagraph{#1}\mbox{}}
\fi
\ifx\subparagraph\undefined\else
\let\oldsubparagraph\subparagraph
\renewcommand{\subparagraph}[1]{\oldsubparagraph{#1}\mbox{}}
\fi

%%% Use protect on footnotes to avoid problems with footnotes in titles
\let\rmarkdownfootnote\footnote%
\def\footnote{\protect\rmarkdownfootnote}

%%% Change title format to be more compact
\usepackage{titling}

% Create subtitle command for use in maketitle
\newcommand{\subtitle}[1]{
  \posttitle{
    \begin{center}\large#1\end{center}
    }
}

\setlength{\droptitle}{-2em}

  \title{UK Household Annual Expenditure Analysis}
    \pretitle{\vspace{\droptitle}\centering\huge}
  \posttitle{\par}
    \author{STAT 420: Final Project, Summer 2019, kamlesh2, kdp3, pa9}
    \preauthor{\centering\large\emph}
  \postauthor{\par}
    \date{}
    \predate{}\postdate{}
  
\usepackage{booktabs}
\usepackage{longtable}
\usepackage{array}
\usepackage{multirow}
\usepackage{wrapfig}
\usepackage{float}
\usepackage{colortbl}
\usepackage{pdflscape}
\usepackage{tabu}
\usepackage{threeparttable}
\usepackage{threeparttablex}
\usepackage[normalem]{ulem}
\usepackage{makecell}
\usepackage{xcolor}

\begin{document}
\maketitle

Loading the libraries required for this analysis

\begin{Shaded}
\begin{Highlighting}[]
\KeywordTok{library}\NormalTok{(tidyverse)}
\KeywordTok{library}\NormalTok{(kableExtra)}
\end{Highlighting}
\end{Shaded}

\hypertarget{introduction-to-the-dataset}{%
\subsection{Introduction to the
dataset}\label{introduction-to-the-dataset}}

The Living Costs and Food Survey (LCF) is an annual survey carried out
in United Kingdom by \href{https://www.ons.gov.uk/}{Office for National
Statistics} since 1957. It collects data on spending pattern and the
cost of living of households across UK.\\
~\\

\includegraphics[width=3.125in,height=\textheight]{https://www.denverlibrary.org/sites/dplorg/files/united_kingdom_map2.jpg}

\hypertarget{data-collection-methodology}{%
\subsubsection{Data Collection
Methodology}\label{data-collection-methodology}}

The LCF sample for Great Britain is a multi-stage stratified random
sample with clustering. Address with `small user' postcodes are drawn
from the postcode address file. The LCF sample for Northern Ireland,
which is part for Great Britain is collected by the central survey unit
of Northern Island Statistics and Research Agency(NISRA). A systematic
random sample of private addresses is drawn from the land and property
service agency property database.

\begin{itemize}
\item
  LCF is a continuous survey which is collected throughout the year to
  ensure seasonal effects are covered.
\item
  Randomely about 11,000 private households are selected each year.
\item
  Since it is completely voluntary, households can choose not to respond
  to the survey.
\item
  Every year on average about 50\% of 11,000 households choose to
  respond to the survey
\item
  Volunteering household needs to fill a Household questionnaire,
  Individual questionnaire and dairy to track the daily expenditure for
  2 weeks for all individuals aged 16 and over.
\end{itemize}

\hypertarget{reason-for-this-survey.}{%
\subsubsection{Reason for this survey.}\label{reason-for-this-survey.}}

The LCF provides information for the Retail Prices Index, National
Accounts estimates of household expenditure, the analysis of the effect
of taxes and benefits and trends in nutrition. The results, however, are
multi-purpose, providing an invaluable supply of economic and social
data.

\hypertarget{lcf-survey-2013}{%
\subsubsection{LCF Survey 2013}\label{lcf-survey-2013}}

The dataset which we are analysing in this project is teaching dataset
which is a subset of LCF 2013 survey. This dataset has been simplified
for the purpose of learning and teaching. This dataset has been
anonymised and deposited with the \href{https://ukdataservice.ac.uk}{UK
data service} and can be found
\href{https://beta.ukdataservice.ac.uk/datacatalogue/studies/study?id=7932\#!/details}{here}.

LCF 2013 survey has 5,144 respondents out of which 151 were from
northern Ireland.\\
~\\

\hypertarget{references}{%
\paragraph{References}\label{references}}

\emph{University of Manchester, Cathie Marsh Institute for Social
Research (CMIST), UK Data Service, Office for National Statistics.
(2019). Living Costs and Food Survey, 2013: Unrestricted Access Teaching
Dataset. {[}data collection{]}. 2nd Edition. Office for National
Statistics, {[}original data producer(s){]}. Office for National
Statistics. SN: 7932, \url{http://doi.org/10.5255/UKDA-SN-7932-2}}\\
~\\

\hypertarget{purpose-of-study}{%
\subsubsection{Purpose of study}\label{purpose-of-study}}

As part of this study, we would like to infer the household expenditure
based on income and other socioeconomic indicators of the household.

We are also interested in knowing if a regression model can be build to
predicti the expenditure with good confidence.

\hypertarget{variable-definitions}{%
\subsection{Variable definitions}\label{variable-definitions}}

The variables in the dataset are

\begin{longtable}[]{@{}lll@{}}
\toprule
\begin{minipage}[b]{0.14\columnwidth}\raggedright
Variable name\strut
\end{minipage} & \begin{minipage}[b]{0.64\columnwidth}\raggedright
Variable label\strut
\end{minipage} & \begin{minipage}[b]{0.14\columnwidth}\raggedright
Variable type\strut
\end{minipage}\tabularnewline
\midrule
\endhead
\begin{minipage}[t]{0.14\columnwidth}\raggedright
casenew\strut
\end{minipage} & \begin{minipage}[t]{0.64\columnwidth}\raggedright
Randomly generated case number\strut
\end{minipage} & \begin{minipage}[t]{0.14\columnwidth}\raggedright
Scalar\strut
\end{minipage}\tabularnewline
\begin{minipage}[t]{0.14\columnwidth}\raggedright
weighta\strut
\end{minipage} & \begin{minipage}[t]{0.64\columnwidth}\raggedright
Annual weight\strut
\end{minipage} & \begin{minipage}[t]{0.14\columnwidth}\raggedright
Scalar\strut
\end{minipage}\tabularnewline
\begin{minipage}[t]{0.14\columnwidth}\raggedright
P550tpr\strut
\end{minipage} & \begin{minipage}[t]{0.64\columnwidth}\raggedright
Total expenditure, by adults \& children (top-coded)\strut
\end{minipage} & \begin{minipage}[t]{0.14\columnwidth}\raggedright
Scalar\strut
\end{minipage}\tabularnewline
\begin{minipage}[t]{0.14\columnwidth}\raggedright
P344pr\strut
\end{minipage} & \begin{minipage}[t]{0.64\columnwidth}\raggedright
Gross normal weekly household income (top-coded)\strut
\end{minipage} & \begin{minipage}[t]{0.14\columnwidth}\raggedright
Scalar\strut
\end{minipage}\tabularnewline
\begin{minipage}[t]{0.14\columnwidth}\raggedright
P425r\strut
\end{minipage} & \begin{minipage}[t]{0.64\columnwidth}\raggedright
Main source of household income\strut
\end{minipage} & \begin{minipage}[t]{0.14\columnwidth}\raggedright
Nominal\strut
\end{minipage}\tabularnewline
\begin{minipage}[t]{0.14\columnwidth}\raggedright
A172\strut
\end{minipage} & \begin{minipage}[t]{0.64\columnwidth}\raggedright
Internet connection in household\strut
\end{minipage} & \begin{minipage}[t]{0.14\columnwidth}\raggedright
Nominal\strut
\end{minipage}\tabularnewline
\begin{minipage}[t]{0.14\columnwidth}\raggedright
A093r\strut
\end{minipage} & \begin{minipage}[t]{0.64\columnwidth}\raggedright
Economic position of Household Reference Person\strut
\end{minipage} & \begin{minipage}[t]{0.14\columnwidth}\raggedright
Nominal\strut
\end{minipage}\tabularnewline
\begin{minipage}[t]{0.14\columnwidth}\raggedright
A094r\strut
\end{minipage} & \begin{minipage}[t]{0.64\columnwidth}\raggedright
NS-SEC 3 class of Household Reference Person\strut
\end{minipage} & \begin{minipage}[t]{0.14\columnwidth}\raggedright
Nominal\strut
\end{minipage}\tabularnewline
\begin{minipage}[t]{0.14\columnwidth}\raggedright
A121r\strut
\end{minipage} & \begin{minipage}[t]{0.64\columnwidth}\raggedright
Tenure type\strut
\end{minipage} & \begin{minipage}[t]{0.14\columnwidth}\raggedright
Nominal\strut
\end{minipage}\tabularnewline
\begin{minipage}[t]{0.14\columnwidth}\raggedright
SexHRP\strut
\end{minipage} & \begin{minipage}[t]{0.64\columnwidth}\raggedright
Sex of Household Reference Person\strut
\end{minipage} & \begin{minipage}[t]{0.14\columnwidth}\raggedright
Nominal\strut
\end{minipage}\tabularnewline
\begin{minipage}[t]{0.14\columnwidth}\raggedright
A049r\strut
\end{minipage} & \begin{minipage}[t]{0.64\columnwidth}\raggedright
Number of persons in household\strut
\end{minipage} & \begin{minipage}[t]{0.14\columnwidth}\raggedright
Ordinal\strut
\end{minipage}\tabularnewline
\begin{minipage}[t]{0.14\columnwidth}\raggedright
G018r\strut
\end{minipage} & \begin{minipage}[t]{0.64\columnwidth}\raggedright
Number of adults in household\strut
\end{minipage} & \begin{minipage}[t]{0.14\columnwidth}\raggedright
Ordinal\strut
\end{minipage}\tabularnewline
\begin{minipage}[t]{0.14\columnwidth}\raggedright
G019r\strut
\end{minipage} & \begin{minipage}[t]{0.64\columnwidth}\raggedright
Number of children in household\strut
\end{minipage} & \begin{minipage}[t]{0.14\columnwidth}\raggedright
Ordinal\strut
\end{minipage}\tabularnewline
\begin{minipage}[t]{0.14\columnwidth}\raggedright
Gorx\strut
\end{minipage} & \begin{minipage}[t]{0.64\columnwidth}\raggedright
Government Office Region - modified\strut
\end{minipage} & \begin{minipage}[t]{0.14\columnwidth}\raggedright
Nominal\strut
\end{minipage}\tabularnewline
\begin{minipage}[t]{0.14\columnwidth}\raggedright
weightar\strut
\end{minipage} & \begin{minipage}[t]{0.64\columnwidth}\raggedright
Weight (rescaled)\strut
\end{minipage} & \begin{minipage}[t]{0.14\columnwidth}\raggedright
Scalar\strut
\end{minipage}\tabularnewline
\begin{minipage}[t]{0.14\columnwidth}\raggedright
maininc\strut
\end{minipage} & \begin{minipage}[t]{0.64\columnwidth}\raggedright
Main source of household income (recoded, P425-1)\strut
\end{minipage} & \begin{minipage}[t]{0.14\columnwidth}\raggedright
Nominal\strut
\end{minipage}\tabularnewline
\begin{minipage}[t]{0.14\columnwidth}\raggedright
income\strut
\end{minipage} & \begin{minipage}[t]{0.64\columnwidth}\raggedright
Income\strut
\end{minipage} & \begin{minipage}[t]{0.14\columnwidth}\raggedright
Scalar\strut
\end{minipage}\tabularnewline
\begin{minipage}[t]{0.14\columnwidth}\raggedright
expenditure\strut
\end{minipage} & \begin{minipage}[t]{0.64\columnwidth}\raggedright
Total expenditure (top coded, formerly P550tpr)\strut
\end{minipage} & \begin{minipage}[t]{0.14\columnwidth}\raggedright
Scalar\strut
\end{minipage}\tabularnewline
\begin{minipage}[t]{0.14\columnwidth}\raggedright
hhsize\strut
\end{minipage} & \begin{minipage}[t]{0.64\columnwidth}\raggedright
Household size, number of people in household (recoded)formerly
A049r\strut
\end{minipage} & \begin{minipage}[t]{0.14\columnwidth}\raggedright
Nominal\strut
\end{minipage}\tabularnewline
\bottomrule
\end{longtable}

The dataset contains 5,144 observations with each observation containing
data from one household. The goal of this project is to build a model to
predict expenditure of a family from the other variables in the dataset.
We will first study and visualise the original data and examine
potential outliers. We visualise and analyse the data and extract
features as we go forward to generate a model and provide visuals.\\
~\\

\hypertarget{loading-the-dataset-in-r}{%
\subsection{Loading the dataset in R}\label{loading-the-dataset-in-r}}

Loading data from the tab delimited file:

\begin{Shaded}
\begin{Highlighting}[]
\NormalTok{lcf_data =}\StringTok{ }\KeywordTok{read.table}\NormalTok{(}\StringTok{"icfforworkbook.tab"}\NormalTok{, }\DataTypeTok{sep =} \StringTok{"}\CharTok{\textbackslash{}t}\StringTok{"}\NormalTok{, }\DataTypeTok{header =} \OtherTok{TRUE}\NormalTok{)}
\end{Highlighting}
\end{Shaded}

Sample data:

\begin{Shaded}
\begin{Highlighting}[]
\KeywordTok{head}\NormalTok{(lcf_data) }\OperatorTok\StringTok{ }
\StringTok{  }\KeywordTok{kable}\NormalTok{() }\OperatorTok\StringTok{ }\KeywordTok{kable_styling}\NormalTok{(}\KeywordTok{c}\NormalTok{(}\StringTok{"striped"}\NormalTok{,}\StringTok{"bordered"}\NormalTok{))}
\end{Highlighting}
\end{Shaded}

\begin{table}[H]
\centering
\begin{tabular}{r|r|r|r|r|r|r|r|r|r|r|r|r|r|r|r|r|r|r}
\hline
casenew & weighta & P550tpr & P344pr & P425r & A172 & A093r & A094r & A121r & SexHRP & A049r & G018r & G019r & Gorx & weightar & maininc & income & expenditure & hhsize\\
\hline
14999 & 2.834724 & 380.6958 & 465.36 & 1 & 1 & 4 & 3 & 1 & 2 & 4 & 2 & 3 & 4 & 0.5432482 & 0 & 465.36 & 380.6958 & 4\\
\hline
14993 & 10.077947 & 546.4134 & 855.26 & 1 & 1 & 1 & 4 & 3 & 2 & 2 & 2 & 1 & 7 & 1.9313442 & 0 & 855.26 & 546.4134 & 2\\
\hline
14993 & 4.088677 & 242.1890 & 160.96 & 1 & 1 & 1 & 2 & 3 & 2 & 1 & 1 & 1 & 8 & 0.7835566 & 0 & 160.96 & 242.1890 & 1\\
\hline
14984 & 4.571760 & 421.3824 & 656.22 & 1 & 1 & 1 & 3 & 3 & 1 & 2 & 2 & 1 & 6 & 0.8761349 & 0 & 656.22 & 421.3824 & 2\\
\hline
14981 & 7.349617 & 370.4056 & 398.80 & 1 & 1 & 1 & 4 & 3 & 1 & 1 & 1 & 1 & 8 & 1.4084853 & 0 & 398.80 & 370.4056 & 1\\
\hline
14976 & 5.046701 & 172.3972 & 321.02 & 1 & 2 & 1 & 3 & 1 & 1 & 1 & 1 & 1 & 11 & 0.9671531 & 0 & 321.02 & 172.3972 & 1\\
\hline
\end{tabular}
\end{table}

The structure of the dataframe is:

\begin{Shaded}
\begin{Highlighting}[]
\KeywordTok{str}\NormalTok{(lcf_data)}
\end{Highlighting}
\end{Shaded}

\begin{verbatim}
## 'data.frame':    5144 obs. of  19 variables:
##  $ casenew    : int  14999 14993 14993 14984 14981 14976 14975 14972 14971 14970 ...
##  $ weighta    : num  2.83 10.08 4.09 4.57 7.35 ...
##  $ P550tpr    : num  381 546 242 421 370 ...
##  $ P344pr     : num  465 855 161 656 399 ...
##  $ P425r      : int  1 1 1 1 1 1 2 1 1 2 ...
##  $ A172       : int  1 1 1 1 1 2 1 1 1 1 ...
##  $ A093r      : int  4 1 1 1 1 1 4 4 1 4 ...
##  $ A094r      : int  3 4 2 3 4 3 5 5 2 5 ...
##  $ A121r      : int  1 3 3 3 3 1 3 1 3 3 ...
##  $ SexHRP     : int  2 2 2 1 1 1 1 2 1 1 ...
##  $ A049r      : int  4 2 1 2 1 1 2 5 2 2 ...
##  $ G018r      : int  2 2 1 2 1 1 2 4 2 2 ...
##  $ G019r      : int  3 1 1 1 1 1 1 1 1 1 ...
##  $ Gorx       : int  4 7 8 6 8 11 12 8 10 3 ...
##  $ weightar   : num  0.543 1.931 0.784 0.876 1.408 ...
##  $ maininc    : int  0 0 0 0 0 0 1 0 0 1 ...
##  $ income     : num  465 855 161 656 399 ...
##  $ expenditure: num  381 546 242 421 370 ...
##  $ hhsize     : int  4 2 1 2 1 1 2 5 2 2 ...
\end{verbatim}


\end{document}
